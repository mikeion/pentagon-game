\documentclass{article}
\usepackage[margin=1in]{geometry}
\renewcommand{\baselinestretch}{1}

\usepackage[utf8]{inputenc}
\usepackage[mathscr]{eucal}
\usepackage{amssymb}
\usepackage{amsmath}
\usepackage{amsthm}
\usepackage{bm}
\usepackage{url}
\usepackage{arydshln}

\usepackage{pgf,tikz}
\usetikzlibrary{arrows,matrix}
\usetikzlibrary{arrows.meta,math}

\theoremstyle{plain}
\newtheorem{thm}{Theorem}[section]
\newtheorem{lem}[thm]{Lemma}
\newtheorem{prop}[thm]{Proposition}
\newtheorem{cor}[thm]{Corollary}
\newtheorem{claim}[thm]{Claim}
\newtheorem{fact}[thm]{Fact}
\newtheorem{open}[thm]{Question}
\newtheorem{exa}[thm]{Example}
\newtheorem{rem}[thm]{Remark}

\newcommand{\Or}{\mathcal{O}}


% Feel free to make a version of this command for other notes
%%%%%%%%%%%%%%%%%%%%%%%%%%%%%%%%%%%
\newcommand{\anote}[1]{\par 
  \framebox{\begin{minipage}[c]{0.95 \textwidth}\color{blue} ALEX'S NOTE:
      #1 \color{black}\end{minipage}}\par}

\newcommand{\mnote}[1]{\par 
  \framebox{\begin{minipage}[c]{0.95 \textwidth}\color{orange} MIKE'S NOTE:
      #1 \color{black}\end{minipage}}\par}
%%%%%%%%%%%%%%%%%%%%%%%%%%%%%%%%%%%%


\title{Chip-Firing and the Sandpile Group of the R10 Matroid}
\author{}
\date{}

\begin{document}

\maketitle

\anote{Here's a note from me.}
\mnote{And here's a note from you! Feel free to change the color of your notes from orange to something else.} 
\section{Introduction}

\[M = \begin{bmatrix}
    1 & 0 & 0 & 0 & 0 &   -1 & 1 & 0 & 0 & 1\\
    0 & 1 & 0 & 0 & 0 &   1 & -1 & 1 & 0 & 0\\
    0 & 0 & 1 & 0 & 0 &   0 & 1 & -1 & 1 & 0\\
    0 & 0 & 0 & 1 & 0 &   0 & 0 & 1 & -1 & 1\\
    0 & 0 & 0 & 0 & 1 &   1 & 0 & 0 & 1 & -1\\
    -1 & 1 & 0 & 0 & 1 &  -1 & 0 & 0 & 0 & 0\\
    1 & -1 & 1 & 0 & 0 &  0 & -1 & 0 & 0 & 0\\
    0 & 1 & -1 & 1 & 0 &  0 & 0 & -1 & 0 & 0\\
    0 & 0 & 1 & -1 & 1 &  0 & 0 & 0 & -1 & 0\\
    1 & 0 & 0 & 1 & -1 &  0 & 0 & 0 & 0 & -1\\
\end{bmatrix}\]

\[M^{-1} = \frac 16\begin{bmatrix}
    3 & 1 & -1 & -1 & 1 &   -1 & 1 & 1 & 1 & 1\\
    1 & 3 & 1 & -1 & -1 &   1 & -1 & 1 & 1 & 1\\
    -1 & 1 & 3 & 1 & -1 &   1 & 1 & -1 & 1 & 1\\
    -1 & -1 & 1 & 3 & 1 &   1 & 1 & 1 & -1 & 1\\
    1 & -1 & -1 & 1 & 3 &   1 & 1 & 1 & 1 & -1\\
    -1 & 1 & 1 & 1 & 1 &   -3 & -1 & 1 & 1 & -1\\
    1 & -1 & 1 & 1 & 1 &   -1 & -3 & -1 & 1 & 1\\
    1 & 1 & -1 & 1 & 1 &   1 & -1 & -3 & -1 & 1\\
    1 & 1 & 1 & -1 & 1 &   1 & 1 & -1 & -3 & -1\\
    1 & 1 & 1 & 1 & -1 &   -1 & 1 & 1 & -1 & -3\\
\end{bmatrix}\]

\[\overline{M} = \begin{bmatrix}
    1 - i & i & 0 & 0 & i \\
    i & 1 - i & i & 0 & 0 \\
    0 & i & 1 - i & i & 0 \\
    0 & 0 & i & 1 - i & i \\
    i & 0 & 0 & i & 1 - i \\
\end{bmatrix}\]
 Multiplying the following matrix by any position should give the moves required to get to the identity.
\[\overline{M}^{-1} = \frac16\begin{bmatrix}
    3+i & 1-i & -1-i & -1-i & 1-i \\
    1-i & 3+i & 1-i & -1-i & -1-i \\
    -1-i & 1-i & 3+i & 1-i & -1-i \\
    -1-i & -1-i & 1-i & 3+i & 1-i \\
    1-i & -1-i & -1-i & 1-i & 3+i \\
\end{bmatrix}\]
\end{document}